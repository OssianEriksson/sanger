%READ ONLY LÄNK
%https://www.overleaf.com/read/xndwjtmxmdnz
% ---------------- %%% A4 konvertering %%% ----------------- %
% ---------------- %%%    AVANCERAT    %%% ----------------- %
\documentclass[12pt]{article}
\usepackage[portrait,a4paper,margin=0cm]{geometry}
\usepackage[utf8]{inputenc}
\usepackage[T1]{fontenc}
\usepackage[swedish]{babel}
\usepackage{pdfpages}
\usepackage{filecontents}
\begin{filecontents*}{config.tex}
% ---------------------------------------------------------- %


% ---------------- %%% FORMATERING %%% ----------------- %

\documentclass[12pt]{article}
%% Import %%
\input{pre.tex}
%% --- %%


% -------------- %%% EGNA INSTÄLLNINGAR %%% -------------- %

%% Om något värde ej önskas, lämna måsvingarna tomma %%

%% Storlek på logga %%
%% Om ingen logga önskas, radera logo.png %%
\newcommand{\logga}
{0.8\textwidth}
%% --- %%

%% Titel %%
\newcommand{\titel}
{Titel}
%% --- %%

%% Värdar/Toastmasters %%
\newcommand{\host}
{Värd}
%% --- %%

%% Meny %%
\newcommand{\forratt}
{Förrätt}
\newcommand{\huvudratt}
{Huvudrätt}
\newcommand{\efterratt}
{Efterrätt}
%% --- %%

%% Datum %%
\newcommand{\datum}
{den \today}
%% --- %%

%% Melodirad %%
% Exkludera melodirad (avkommentera om du vill att melodiraden ska gömmas)
%\renewcommand{\mel}[1]{} % Ingen melodirad
%% --- %%

%% Repristecken %%
% Avkommentera båda om du vill byta ut repristecknet mot textekvivalent
%\renewcommand{\repopen}{//:}
%\renewcommand{\repclose}{://}
%% --- %%

% ------------------- %%% DOKUMENT %%% ------------------- %

\begin{document}
\titelsida
\inputsong{livet}
\inputsong{porthos}
\inputsong{vikingen}
\inputsong{fetafransyskor}
\inputsong{bordeaux}
\inputsong{strejkpapripps}
\inputsong{gamlakall}
\newpage
\inputsong{sistapunschvisan}
\end{document}

% -------------------------------------------------------- %


























% ---------------- %%%     RÖR EJ      %%% ----------------- %
\end{filecontents*}

\makeatletter
\def\app@exe{\immediate\write18} % Definiera körbara Linux-kommandon
\def\createBooklet{%
    \app@exe{pdflatex config.tex}% Kompilera config.tex som har skapats genom filecontents-paketet

    % Räkna antalet sidor
    \pdfximage{config.pdf}
    
    % Väljer automatiskt rätt layout för antalet sidor
    
    % 1-2
    \ifnum\the\pdflastximagepages=1
        \includepdf[pages=-,nup=2x1]{config.pdf}
    \fi
    \ifnum\the\pdflastximagepages=2
        \includepdf[pages=-,nup=2x1]{config.pdf}
    \fi
    
    % 3-4
    \ifnum\the\pdflastximagepages=3
        \includepdf[pages={{},1,2,3},nup=2x1]{config.pdf}
    \fi
    \ifnum\the\pdflastximagepages=4
        \includepdf[pages={4,1,2,3},nup=2x1]{config.pdf}
    \fi
    
    % 5-8
    \ifnum\the\pdflastximagepages=5
        \includepdf[pages={{},1,2,{},{},3,4,5},nup=2x1]{config.pdf}
    \fi
    \ifnum\the\pdflastximagepages=6
        \includepdf[pages={{},1,2,{},6,3,4,5},nup=2x1]{config.pdf}
    \fi
    \ifnum\the\pdflastximagepages=7
        \includepdf[pages={{},1,2,7,6,3,4,5},nup=2x1]{config.pdf}
    \fi
    \ifnum\the\pdflastximagepages=8
        \includepdf[pages={8,1,2,7,6,3,4,5},nup=2x1]{config.pdf}
    \fi
    
    % 9-12
    \ifnum\the\pdflastximagepages=9
        \includepdf[pages={{},1,2,{},{},3,4,9,8,5,6,7},nup=2x1]{config.pdf}
    \fi
    \ifnum\the\pdflastximagepages=10
        \includepdf[pages={{},1,2,{},10,3,4,9,8,5,6,7},nup=2x1]{config.pdf}
    \fi
    \ifnum\the\pdflastximagepages=11
        \includepdf[pages={{},1,2,11,10,3,4,9,8,5,6,7},nup=2x1]{config.pdf}
    \fi
    \ifnum\the\pdflastximagepages=12
        \includepdf[pages={12,1,2,11,10,3,4,9,8,5,6,7},nup=2x1]{config.pdf}
    \fi
    
    % 13-16
    \ifnum\the\pdflastximagepages=13
        \includepdf[pages={{},1,2,{},{},3,4,13,12,5,6,11,10,7,8,9},nup=2x1]{config.pdf}
    \fi
    \ifnum\the\pdflastximagepages=14
        \includepdf[pages={{},1,2,{},14,3,4,13,12,5,6,11,10,7,8,9},nup=2x1]{config.pdf}
    \fi
    \ifnum\the\pdflastximagepages=15
        \includepdf[pages={{},1,2,15,14,3,4,13,12,5,6,11,10,7,8,9},nup=2x1]{config.pdf}
    \fi
    \ifnum\the\pdflastximagepages=16
        \includepdf[pages={16,1,2,15,14,3,4,13,12,5,6,11,10,7,8,9},nup=2x1]{config.pdf}
    \fi
    
}
\makeatother

\begin{document}
\createBooklet
\end{document}
